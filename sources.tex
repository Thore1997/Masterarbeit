\section {Definition Outlier Analysis}

What are Outliers?
(S20 S.2)	Outliers are sometimes useless and sometimes do they provide usefull information.
			Outliers deviate extremely from well defined norms.
			In some cases we would rather remove them as they mislead our analysis, while in other cases they could 
			they could be very useful and keeping them would be the best solution.
			E.g. Measurment or recoding errors. exceptional but true values, mis-reporting, sampling error
(H80 S.1)	"An observation wich deviates so much from other observations as to arouse that it was generated by a different mechanism"
(C09 S.1)	Outliers and Anomalies are often been used interchangeably

What is Noise?
(S20 S.2)	Noisy data is data wich we have no benefit. Data that carries only meaningless information
			E.g. Incorrect data type (string type entered for a numeric attrtiute), Erroneaous data values
			Missing values (unrecorded data for an attribute)
(C09 S.3)	Noise removal is driven by the need to remove the unwanted objects before any data analysis is performed

What is noise accommodation?


Are there different outlier types?
(C09 S.7)	Point Anomalities: Simplest type of anomaly. 
			If an indivisual data instance can be considered anomalous with respect to the rest of data, 
			then the instance is termed a point anomaly.
(C09 S.7-8)	Contextual Anomalies also conditional (S07 S?): If a data instance is anomalous in a specific context, but not otherwise.
			Each data instance is defined using two sets of attributes: 
			contextual attributes: are used to determine the context (neighborhood) fo that instance.
			E.g. in times series data, time is a contextual attribute that determines the position of an instance
			on the entire sequence
			Behavioral attributes: define the noncontextual characteristics of an instance. 
			In spartial data sets of rainfall of the entire world, the amoutn of rainfall at any location is behavioral attributes
			A Data instance might be a contextual anomaly in a given context, but an identical data instance could be considered normal in a different instance.
			E.g. A similar example can be found in the credit card fraud detection domain. A con-
			textual attribute in the credit card domain can be the time of purchase. Suppose an
			individual usually has a weekly shopping bill of 100 except during the Christmas
			week, when it reaches 1000. A new purchase of 1000 in a week in July will be con-
			sidered a contextual anomaly, since it does not conform to the normal behavior of the
			individual in the context of time even though the same amount spent during Christmas
			week will be considered normal

(C09 S.8-9)	Collective Anomalities: The individual data instance in a collective anomyl may not be anomaliess by themselves, but their 
			occurance together as a collection is anomalous. Is a collection of related data instances is anomalous wih 
			respect to the entire data set. It can only occur in data sets in wich data instances are related.

			A point anomaly or a collective anomaly can also be a contextual anomaly if
			analyzed with respect to a context. Thus a point anomaly detection problem or collec-
			tive anomaly detection problem can be transformed to a contextual anomaly detection
			problem by incorporating the context information.





















