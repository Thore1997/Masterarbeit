\documentclass[a4paper, 	% Seitenformat
		12pt, 								% Schriftgr��e
		bibliography=totoc, 	% Literaturverzeichnis in das Inhaltsverzeichnis
		index=totoc, 	% Index in das Inhaltsverzeichnis
		abstract=true, 	% mit Abstrakt
		headsepline, 	% Trennlinie f�r die Kopfzeile
		%footnosepline,	% Trennlinie f�r die Fusszeile
		]{scrartcl}

\usepackage[T1]{fontenc}
\usepackage[utf8]{inputenc}
\usepackage{lmodern}
\usepackage[english]{babel}
%\usepackage{ulem}
% Bilder
%\usepackage{pst-all}		% Zeichnungen in Latex (kein pdflatex)
\usepackage{subfig}	  		% Unterteilungen von Abbildungen
\usepackage{float}		  	% Gleitobjekte
\usepackage{graphicx}		% Einbinden von Graphiken aus Dateien
\usepackage{rotating}   	% Die folgenden Packages sind meist hilfreich f�r das Erstellen von Arbeiten
\usepackage{mathrsfs}
\usepackage{eurosym,bm,amsmath,amssymb,verbatim}
\usepackage{listings}
\usepackage[T1]{fontenc}
\usepackage{setspace}
\usepackage [round,authoryear] {natbib} % Notwendiges Package f�r den verwendeten Literaturstil
%\usepackage{natbib}
\usepackage{xcolor}
\usepackage[colorinlistoftodos,prependcaption,textsize=tiny,textwidth=2cm]{todonotes}
%------------------------------------------------------------------------

\parindent0em 					% kein Einzug nach einer Leerzeile
\setlength{\parskip}{1em}			
\topmargin-20.4mm   			% F�r die R�nder
\textheight245mm
\textwidth157mm \oddsidemargin11mm \evensidemargin11mm
\hyphenpenalty=100
\exhyphenpenalty=100

%%%%%%%%%%%%%%%%%%%%%%%%%%%%%%%%%%%%%%%%%%%%%%%%%%%%%%%%%%%%%%%%%%%%%%%
%%%%%%%%%%%%%%%%%%%%%%%%%%%%%%%%%%%%%%%%%%%%%%%%%%%%%%%%%%%%%%%%%%%%%%%
%%%%%%%%%%%%%%%%%%%%%%%%%%%%%%%%%%%%%%%%%%%%%%%%%%%%%%%%%%%%%%%%%%%%%%%

\begin{document}					% Start des Dokuments
\begin{titlepage}

\hspace{12cm}\scalebox{0.3}{\includegraphics{Unilogo.jpg}}

\vspace*{1cm}
\large\textbf{Disposition: }\\[0.5cm]
\large\textbf{From Statistics to Machine Learning: \\ A Comparative Study of Outlier Detection Methods \\ in Multidimensional Data}\\[0.5cm]

Studiengang: Betriebswirtschaftslehre\\

Lehrstuhl für Statistik\\[1.5cm]
%
%
\noindent\begin{tabular}[h]{@{}ll}
%\begin{tabular}{l l}
Eingereicht bei: &Prof. Dr. Yarema Okhrin \\
\\
Betreuer: &Herr Philipp Haid 			 \\
\\
Vorgelegt von:   &Thore Johannsen         \\
\\
Adresse: &Salomon-Idler-Straße 4         \\[0.2cm]
                 &Augsburg                      \\
\\
Matrikel-Nr.:    &2231165                 \\
\\
Email:           &thore.johannsen@outlook.com     \\
\\
\end{tabular}
\vfill
Augsburg, im September 2010
\end{titlepage}
\newpage
\pagenumbering{Roman}


\tableofcontents		    % Inhaltsverzeichnis
\newpage
%\listoffigures			    % Abbildungsverzeichnis
%\listoftables				  % Tabellenverzeichnis

\clearpage
\pagenumbering{arabic}
\onehalfspacing 			     % F�r richtigen Zeilenabstand

%%%%%%%%%%%%%%%%%%%%%%%%%%%%%%%%%%%%%%%%%%%%%%%%%%%%%%%%%%%%%%%%%%%%%%%
%%%%%%%%%%%%%%%%%%%%%%%%%%%%%%%%%%%%%%%%%%%%%%%%%%%%%%%%%%%%%%%%%%%%%%%
%%%%%%%%%%%%%%%%%%%%%%%%%%%%%%%%%%%%%%%%%%%%%%%%%%%%%%%%%%%%%%%%%%%%%%%

This thesis undertakes a comparative performance analysis of three outlier detection methods: the Mahalanobis distance using FAST-MCD algorithm \citep{rousseeuw1999fast}, the k-Nearest Neighbors (kNN) approach \citep{ramaswamy2000efficient}, and the contrastive learning–based method proposed by \citet{shenkar2022anomaly}.
Outlier detection constitutes an essential component across scientific research and industrial applications. \\
The presence of anomalous data points can compromise statistical inference. 
Specifically, outliers possess the capacity to inflate estimates of variability (e.g. variance and standard deviation), consequently widening confidence intervals. 
This effect increases the probability of committing a Type II error (failure to reject a false null hypothesis). 
Conversely, in certain domains, the outliers themselves represent the primary interest. 
Prominent examples include the application in financial fraud detection \citep{ramaswamy2000efficient} and in medical diagnostics for identifying pathological conditions, such as cervical cancer \citep{ijaz2020data}. \\
The primary objective of this investigation is to systematically evaluate the relative efficacy of the three selected algorithms in identifying outliers within a designated dataset.
Following the formal definition and characterization of outliers, the study will detail the underlying mechanisms of the three analyzed methods. 
This paper will begin with the robust statistical method of FAST-MCD. 
Subsequently, the analysis will proceed to discuss the distance-based method of kNN. 
Finally, the Intercont approach, a novel machine learning method will be presented.
Different performance metrics will be discussed to supply a optimal view of the detection performance. 
The empirical investigation will utilize financial datasets, drawing among other things primary research literature.
The study is guided by the following formal hypotheses regarding detection efficacy:
The kNN algorithm is hypothesized to yield the highest anomaly detection performance among the three candidates.
The contrastive learning–based Intercont algorithm is anticipated to outperform the FAST-MCD method but is expected to demonstrate lower efficacy than the kNN algorithm.

  % Zur besseren Strukturierung empfiehlt es sich das Dokument aufzuteilen.
\newpage
\section{Context}
Outlier Analysis is an important task for research and real world applications like spam filters.
Suspicious observations should be marked as Outliers and inliners should not be marked as outliers. 
Nevertheless, a uniform definition of outliers could yet not be found. In 2017 \citet{ayadi2017outlier} found 12 different interpretations of outlier.
All of them dont differ so much, so I will use the definition of Hawkins, because of the influece his paper had on this topic.
Ouliers, also called anomalies \citep{chandola2009anomaly}, are observations, that deviate extremely from the defined norm.
They are the product of a different mechanisms, that aren't included into the research question. \citep{hawkins1980identification}
On the one hand, depending on the context they can be researchers target and provide useful information.
For instance outliers develope by measuring recording errors, misreporting or sample errors.
E.g. in financial domains, outliers are been used to identify criminal activity like creditcard fraud. 
The usage of the stolen credit card by the criminal should be very different of the usage of the rightful credit card holder. \citep{rezapour2019anomaly}
On the other hand, if they are not the target, they have to be identified and carefully treated. \citep{smiti2020critical}
You can treat them by removing, transforming, winzoriesing or using robust alternatives.
Every method has its advantages and disadvantages, but it would be out of scope of this thesis to go deeper into it. 


Different from outliers is noisy data.
Noisy data has no useful information and hinders the analysis. 
It need to be identified and corrected because it makes computation difficult or impossible to do.
Noisy data contains incorrect data types, missing values or incorrect data \citep{smiti2020critical}.
There are 3 different types of outliers: 


\begin{enumerate}
\item Global outliers, sometimes named point anomalies, are the simplest type of anomaly. 
When a obervation with respect to rest of the its distribution has extreme values, you could consider it a outlier.
E.g. a Person spends 100€ a week and someday the person spends 1000€. that would be possibly be a global outlier, because it is 
a extreme value out of the normal distribution. 
\item Contextual outliers, sometimes named conditional outliers, are outliers that are outliers bescause of their specific contexts.
A data instance is defined by it behavioral and contextual attributes. 
Contextual attributes define contextual atrributes and behavioral attributes define non-contextual attributes.
Using the credit card fraud example, behavioral atrributes are the amount of paymets done in a certain period.
Contextual attribute is the time dimension of this longitudinal data. 
If a person spends 100€ in a week, accept for christmas holiday. There the Person spends 1000€.
the contextual outlier would be, if a transaction of 1000€ emerges in mid july.
1000€ is not an extreme charakteristic, because it will be spend in a different time.
But it is a unusual period of time. 
\item Collective outliers are observations that are by itselfe not unsual but their accurance together as a collection. 
Using the example as before, a single use of a credit card for a normal payment is not a outlier, but when a person pays 50 times
a small payment, that would be considered a collective outlier.
\end{enumerate}


All 3 are closely related. A global outlier or a collective outlier can also be a contextualized and treated like a 
contextual problem. 


A different, but closley related, type of detection is novelty detection. Often both terms are being used synonymously \citep{domingues2019probabilistic}.
But it targets a different scenario. 
Noveltey detection is been used to find new and previously unconsidered patterns in the data.
The findings are normally incorporated into a new model and not like in outlier detection being removed or eximaned further \citep{chandola2009anomaly}.


A fairly new approach in outlier detection is to provide meaningful explanations for outliers.
There is a need to tell not just wich observation is a outlier, but to tell why that specific outlier is a outlier.
E.g. in Fraud detection, analyst need to understand why a transaction was being flagged as fraudulent and wether it need human intervention.
A common approach is to assign a importnace level to the outlier.
The more far away an outlier is from the "normal" data distribution, the more likely it need human attention 
Therefor a user can prioritize. 


A different type a providing meaningfull insights in outliers to highlight causal interactions among outliers. 
Causal in a sense, that one outlier produced other outliers in chronological way.
So if you remove that first outlier, you would be removing the other outliers too.
For example a traffic jam in one part of the city is beeing caused of another part of the city
Users therefor can use that information to prevent the first outlier, so that other outliers don't develope \citep{panjei2022survey}.
\newpage
\section{Methods for detecting outliers}
There are plentyfull of methods to detect outliers. Each with advantages and disadvantages.
The most popular methods could be categorized as followes: statistical-based, distance-based, density-based,  clustering-based,
graph-based, ensemble-based and learning-based.\citep{wang2019progress} 
In this thesis, I will examine 3 different methods. A statistical, a clustering-based and a learning-based Method.
Statistical methods or distribution-based methods, consider a data point as an outlier, if it extremly deviates from its distribution.
Statisticians used tools like Box-plots, mean-variance or regression models to detect extreme data points \citep[p.3]{smiti2020critical}.



In this section I will begin describing the robust Mahalanobis Distance (MCD)
Therefor I will describe the Minimum Covariance Determinant used in calculating Mahalanobis Distance.
Afterwards I will transition to the FAST-MCD Methods, because that will be used in my thesis.

\subsection{Mahalanobis Distance}
The Mahalanobis Distance is a measurment in multidimensional space for the distance between a vector and its distribution mean.
It considers the correlation between dimensions and calculates more reliable results.
Not like Euclidean distance that just considers two different points, but ignores correlation between Variables. 
But uncorrelated variables are further away from its distribution mean than highly correlated variables. 

The Mahalanobis distance $D_M$ is defined as: 

\begin{equation}
D_M(\mathbf{x}, \mathbf{\mu}) = \sqrt{ (\mathbf{x} - \mathbf{\mu})^T \mathbf{\Sigma}^{-1} (\mathbf{x} - \mathbf{\mu}) }
\label{eq:mahalanobis}
\end{equation}

where as $x$ is the vector of the obervation, $\mu$ is the mean vector and $\Sigma$ is the covariance matrix. 
By multiplying the distance between the vektor and its mean and the inverse of the covariance matrix, we standardize the distances to their correlations.
Resulting in a high Mahalanobis distance for uncorrelated vaiables and and a low $M_D$ for highly correlated variables.

What advantages and disadvantages are there?

What is the solution?
The Minimum Covariance Determinant, first published in 1985 by \citet{article}, is a robust estimator of the covariance matrix in multidimensional dataset.
Given a dataset of n observations and a predetermined subset sie h, the MCD searches for the h observations whose empirical covariance matrix has the smalltest possible determinant.
A determinant is a mathematical property of a matrix.

How does it work?


What are the advantages?

What are the disadvantages?
Minimum Coavariance Determinant is asymptotically normal and robust. But it is hard to compute.

Therefor the authors constructed an algorithm, thats much faster than the first algorithm.
\subsection{FAST-MCD}





\newpage

The Ordinary Least Squared was being critized because of its lack of robusteness. Just one outlier could have a large negative effect.

Mahalanobis disctance cant detect multiple outliers because of the masking effect, by wich multiple outliers dont have necesssarly a large mahananobis disctance (R99 S.212)
Main idea of MCD find h observations out of n whose classical covariance matrix has the lowest determinant. The MCD estimate is then rhe average of these h points 
and the MCD estimate of scatter is their covariance matrix. 

Procedure: 1. Creating initial subset. 
2 Possibillities: draw a random h subset H1, or draw a random (p+1)-subset auf J, and then compute 
technical terms:

Determinant - Sie ist das Produkt der Hauptdiagonalen minus dem Produkt der Nebendiagonalen. Das ist stimmt bis zu einer 3x3 Matrix, darüberhinaus werden andere Methoden verwendet (Laplace'scher Entwicklungssatz)
Es ist eine Eigenschaft die Auskunft über die Volumenverzerrung und Inventierbarkeit einer Matrix gibt. 
Inventierbar ist eine Matrix, wenn det (A) =/ 0. 
Geometrische Bedeutung:  Der Betrag der Determinante ist gleich dem Faktor, um den sich das Volumen (oder Fläche)  eines Einheitswürfels nach der Transformation vergrößert oder verkleinert.
Das Vorzeichen der det(A) gibt Auskunft über die Orientierung. det (A) > 0 bleibt Orientierung erhalten (zb Drehung oder Streckung)
Wenn det(A) < 0 wird die Orientierung umgekehrt (z.B. eine Spiegelung)

Break-down value

asymptotisch normalverteilt 
\newpage
\section{Benchmarking}
\section{Outlier Detection in Dataset}
\newpage
\section{Results}
\section{Discussion}

Try to make a point, that the implementation of new machine learning approaches bring advantages in fraud detection, but the implementation will be so 
costly, that for smaller corporation thats not suitable. Rather they deal with a not so perfect detection method that investing large amounts of capital
to detect a small number of outliers

%\vspace{5em}

%%%%%%%%%%%%%%%%%%%%%%%%%%%%%%%%%%%%%%%%%%%%%%%%%%%%%%%%%%%%%%%%%%%%%%%
%%%%%%%%%%%%%%%%%%%%%%%%%%%%%%%%%%%%%%%%%%%%%%%%%%%%%%%%%%%%%%%%%%%%%%%
%%%%%%%%%%%%%%%%%%%%%%%%%%%%%%%%%%%%%%%%%%%%%%%%%%%%%%%%%%%%%%%%%%%%%%%

Survey about 

The fundamental idea of statistical-based techniques in labeling or identifying outliers depends on the relationship with
the distribution model. These methods are usually classified into two main groups - the parametric and non-parametric
methods. 
The key idea for clustering-based techniques is the application of standard clustering techniques to detect outliers from
given data. Outliers are considered as the observations that are not within or nearby any large or dense clusters.
Learning-based methods such as active learning and deep learning, the underlying idea is to learn different models
through the application of these learning methods to detect outliers (W19 S.107967 ff)


\newpage 


Concept of my Thesis:
My narrative compares a traditional distance-based method (k-NN) with a modern approach Contrasive Learning approach (Intercont). 
MCD fits as the Classical Statistical Benchmark against which both techniques are measured.
Chapter I: The Need for Robustness (MCD as the Theoretical Gold Standard)
I want to introduce MCD as the foundational method developed in the statistical community to solve the problem of multivariate robust estimation.
It represents the gold standard for defining a "true" outlier based on a well-defined theoretical model (multivariate normality).
The key Concept to Highlight is its high breakdown point %\approx 50\%. 
It can resist contamination that would utterly destroy traditional estimators like the sample mean and covariance. (how much contamination can the others hold up to?)
MCD defines what a robust outlier detection method should achieve in terms of resistance to contamination. 
This sets the bar for robustness.


Chapter II: The Limitations (The Gap that my methods fill)
I want to use MCD's weaknesses to establish the necessity of the methods you are actually comparing (k-NN and Modern ML).
I want to shift the debate to the other methods.
Limitations to Discuss:
Computational Bottleneck: MCD struggles with large N and high P.
I want to state clearly that MCD's strength is simultaneously its greatest weakness: it is a parametric method.
It relies on the strong, central assumption that the clean data follows an elliptical distribution (most often multivariate normal). 
This establishes the need for computationally scalable methods.
Model Dependence: MCD is inherently tied to the assumption of a globally elliptical (e.g., multivariate normal) distribution. 
Use MCD's strict assumptions to motivate the need for non-parametric methods.
MCD's Constraint: Real-world datasets (especially in business, IT, or complex sensors) are rarely normally distributed. 
They are often multi-modal or have complex, non-linear structures. MCD is designed to find global outliers relative to a single, central ellipse.
The Thesis Problem: Because real-world data structures are complex, a method designed to find a single, central data "cloud" (MCD) will fail to identify local outliers or outliers
in non-elliptical shapes. 
This failure is what necessitates the shift to your two core methods.
This assumption often fails in complex, real-world data, leading to the need for non-parametric (k-NN) or distribution-agnostic (Modern ML) methods.
Therefor I will analyze real-world data about financial fraud that was used befor in papers.
Thesis Fit: MCD is too slow and too model-dependent for the types of large, complex datasets you are using.
This explains why k-NN and Modern ML are relevant contenders for modern outlier detection.


Chapter III: Introducing k-NN and InterCont
Defining the two methods and describing their assumptions
Whats special about k-NN and what is special about InterCont?
where are weaknesses?


Chapter IV: Benchmark Comparison (MCD as the Control)
Defining Benchmarks and describing short what "exotic" benchmarks are out there.
Role of MCD: Include MCD in your empirical analysis as a control method or baseline benchmark.
The Comparison Strategy:
Data A (Ideal/Synthetic): Test all three methods on a dataset that is perfectly multivariate normal with scattered outliers. 
Here, MCD should ideally perform the best or, at least, as well as the others in terms of detection rate (Area Under the Curve, AUC). 
This validates the theoretical strength of MCD.
Data B (Real/Complex): Test all three methods on a large, complex, non-elliptical dataset (the type of data where k-NN and Modern ML excel). 
Here, MCD should perform poorly or fail to run efficiently due to its limitations, thereby highlighting the superior scalability and flexibility of k-NN and the Modern ML method.
Thesis Fit: By showing MCD's success on ideal data and its failure on real-world data, you clearly demonstrate the shift in necessary methodology, creating a powerful contrast that
elevates the performance of k-NN and Modern ML. Describe the advantages and disadvantages about the other methods as well


Chapter V: Conclusion
Maybe you can argue are ask a follow up question, if such a comlex method as Intercont is economically reasonable.
You need specially trained employees do set it up and take care of.
Maybe you can try to explain why k-NN is still performing good in comparison to InterCont.
The necessary step from statistics to machine learing is clear and you proven it. 
But is it so clear to implement a deep learing method, when a lazy learner method like k-NN can provide good results as well?


\clearpage
\bibliographystyle{autorjahrdidiDE}  % Dies ist der verwendete Stil f�r Zitation und Literaturverzeichnis																	 % Die Datei autorjahrdidiDE.bst muss in das aktuelle Arbeitsverzeichnis kopiert werden
\bibliography{Literatur}	           % Literaturverzeichnis liegt in der Datei Literatur
\end{document}											 % Ende des Dokuments
