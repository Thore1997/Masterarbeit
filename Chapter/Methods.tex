\section{Methods for detecting outliers}
There are plentyfull of methods to detect outliers. Each with advantages and disadvantages.
The most popular methods could be categorized as followes: statistical-based, distance-based, density-based,  clustering-based,
graph-based, ensemble-based and learning-based.\citep{wang2019progress} 
In this thesis, I will examine 3 different methods. A statistical, a clustering-based and a learning-based Method.
Statistical methods or distribution-based methods, consider a data point as an outlier, if it extremly deviates from its distribution.
Statisticians used tools like Box-plots, mean-variance or regression models to detect extreme data points \citep[p.3]{smiti2020critical}.

In this section I will begin describing the robust Mahalanobis Distance and explain why it is adventageous to use the mininum covariance determinant.
Afterwards I will transition to the FAST-MCD Methods, because that will be used in my thesis.
In section 2.2 I will describe the k-nearest neighbor algorithm.
After that I will explain a expansion of k-NN.
In Section 2.3 I will define the newly published Intercont method, wich is based on contranstive learning.


\textcolor{green}
{ 
The curse of dimenionality (CoD) plays a big role in highdimensional datasets.
However, in the big data era, the sheer number of variables that can be collected from a single sample can be
problematic. we discuss four important problems of dimensionality as it applies to data sparsity, multicollinearity, testing and overfitting
First, as the dimensionality p increases, the volume that the samples may occupy grows rapidly. If we treat the distance between points
(e.g., Euclidian distance) as a measure of similarity, then we interpret greater distance as greater dissimilarity. As p increases, this
dissimilarity increases because the mean distance between points increases as wurzelp (Fig. 2a). This effect is stark at high values of
p. For example, at p = 100, the closest pair of points are farther from one another than the most distant two points are for about p < 15
400this month The assessment of whether a particular data value is an outlier accordingly also becomes difficult. For example, 68\% of normally
distributed points in one dimension fall within sigma of the mean, but this fraction drops off precipitously for large values of p (Fig. 2b).
For example, at p = 10, the fraction of points within sigma of the mean is only 0.017\%, which is equivalent to the points within 0.00022sigma in
one dimension (Fig. 2b, dashed lines). Putting it another way, points within sigmaat p = 10 are as rare as points outside of 3.8sigma at p = 1. 
These are not intuitive observations—proportions of how points are distributed at higher dimensions are not the same as in one dimension.
hat if we use a different distance measure to express similarity between subjects, such as correlation? Unfortunately, we cannot escape the CoD—the range of
correlations between points drops rapidly (Fig. 2c) with p. For example, at p = 100, among 10,000 random pairs of points we see no correlations greater than 0.5, and most
are extremely tightly grouped around 0. As the number of variables increases, the number of subjects in each set of categories decreases and the
correlation between any two subjects across variables also decreases.
When the number of dimensions is larger than the number of samples (p > n), another effect that confounds analysis is perfect multicollinearity.
In this case, we can always express at least one of the variables as a linear combination of the others. For example, if we
have p = 3 variables but only n = 2 subjects and we think of the subjects as two 3D vectors, then from linear algebra we know that
these vectors define either a point (i.e., they have the same three values) or a line. In both cases, the three variables are related linearly.
This is the case any time there are fewer samples than dimensions—the variables span a lower-dimensional subspace in which some
of the dimensions become redundant and expressible in terms of other dimensions, thus yielding perfect multiple correlation.
Finally, overfitting is another CoD that occurs because the flexibility of prediction equations5 is in part determined
by the number of variables involved. With increased flexibility, prediction and classification rules adapt to both the
patterns in the population and the random idiosyncrasies of the training sample. \citep[p.1pp]{altman2018curse}
\todo{research: embeddings to solve the curse of dimensionality}
}
    
\subsection{Oulier detection methods based on statistics}
The Mahalanobis Distance is a measurment in multidimensional space for the distance between a vector and its distribution mean.
It considers the correlation between dimensions not like for example Euclidean distance that just considers two different points, but ignores correlation between variables. 
But uncorrelated variables are further away from its distribution mean than highly correlated variables. 
By considering the correlation between variables, it is effective for outlier detection in multidimensional datasets, because 
observations that dont have extreme values in single variables, can be flagged as an outlier when considering all variable in combination. 

The Mahalanobis distance $D_M$ is defined as: 

\begin{equation}
    D_M(\mathbf{x}, \mathbf{\mu}) = \sqrt{ (\mathbf{x} - \mathbf{\mu})^T \mathbf{\Sigma}^{-1} (\mathbf{x} - \mathbf{\mu}) }
    \label{eq:mahalanobis}
\end{equation}

where as $x$ is the vector of the obervation, $\mu$ is the mean vector and $\Sigma$ is the covariance matrix. 
By multiplying the distance between the vektor and its mean and the inverse of the covariance matrix, we standardize the distances to their correlations.
Resulting in a high Mahalanobis distance for uncorrelated vaiables and and a low $M_D$ for highly correlated variables.

The Mahalanobis distance is scale-invariant. In the appendix is a formal proof.
It is not affected by the different scales of features because it measures the distance in units of standard deviations.
So changes in feature units can be made without influencing the resulting distance and there is no need to manually tranform features with different
units to ensure they contribute equally to the distance calculation.

By assuming, that the dataset follows a multivariante normal distribution, you also can assume, that the squared Mahalanobis distance is following a $X^2$ distribution.
You can construct a parametic $X^2$ test to spot outliers with a given significance level ($alpha$). 
Therefor you determine the degrees of freedom, choose a significance level and find the critical $X^2$ value.
If the Mahalanobis distance is bigger than the critical vlue, you can call this observation with a given percentage of uncertainty an outlier. 

Unfortuantely the Mahalanobis distance suffers from a masking effect.
And by simply testing the biggest Mahalanobis distance for outiers, you could get false negative results.
Because it is not robust against outliers itselve.

Both input variables, sample mean and covariance matrix, are highly influenced by outliers.
If a outlier occurs, the sample mean will be draged into the direction of the outlier.
Resulting in an artificial scaled down outlier. 
The covariance matrix describes the shape and orientation of the distribution.
The inflating effect of an outlier, streches the ellipsiod in the direction of the ourlier. 
Resulting in an smaller values in the matrix, because you use the inverse.

To solve the masking effect, plenty of robust Mahalanobis distance calculation were proposed in the literatur.
An estimator is called robust, if an outlier just has a small or no effect on the resulting estimators . 
To asses the robustness of an estimator, \citet{hampel1971general} proposed the breakdown point.
Given a dataset $X$ of size $n$, and an examined location estimator $T_n$, the breakdown value is define as:

\begin{equation}
    \epsilon_{n}^{*}(T_{n}, X_{n}) = \frac{1}{n} \min \left\{ m : \sup_{X_{n},{m}} \left\| T_{n}(X_{n,m}) - T_{n}(X_{n}) \right\| = +\infty \right\}
\end{equation}

where as $m$ is the minimum number of observations that need to be replaced with outliers, $X_m$ is the contaminated dataset where $m$ points from $X$ have been replaced by arbitrary values.
It measures the robustness by determing the smallest fraction of outlier needed, for the estimator to take on arbitrary values.
$epsilon_n^*$ is the smallest fraction that causes this unbounded change. 
The maximum value of $epsilon_n^*$ is 0.4 because if the estimator reaches beyond this boundary, it would imply that the estimator would prefer the uncontaminated smaller portion 
of the dataset. If the contaminted data reaches the majority of the dataset, no estimator can distinguish the true underlying distribution from the contaminated one. 
The result could be wrong, because the estimator could have chosen the contaminated dataset as the core dataset.


\subsubsection*{Minimum Covariance Determinant}

Used for mahalanobis distance the breakdown point would be 0, because it takes just one outlier for the variance matrix and/or mean vector to take on arbitraty values. 
To solve this problem, I will use the minimum covariance determinant (MCD). 
The goal of MCD is to find the subset $H$ of $h$ observations in a given dataset with $n$ observations and $p$ variables that minimizes the determinant of the subset's sample covariance matrix.
The determinant $det(A)$ is a scalar of a sqaure matrix $A$.
It determines among other things  the change in orientation and volume.
The absolut value of the determinant the factor of wich the volume changes by multiply the determinant with its matrix $A$.
If the determinant is negative, additionally to changes in volume, the orientation changes as well.


MCD is affine equivariant. That means that MCD estimations of the mean and covariance will follow accordingly to 
affine transformations. For example in a change of measurement, the data will be roted, translated or rescaled but the resulting 
mean and covariance will transform correctly and not stay the same.
Thats important for outlier detection, because estimated outliers will stay outlier even affine transformation took place. \citet{hubert2018minimum}


MCD estimators can only be calculated when the number of subset observations ($h$) exceeds the number of variables ($p$), becaus if not the determinant 
of the resulting covariance matrix will be 0. \citet{hubert2018minimum} recommends at least five observations per dimension to avaid excessiv noise. 
The subsets $h$ size is lays between $\frac{n}{2} \le h \le n$. Clearly it is not possible for subsets observations $h$ to expand the number of total observations $n$. 
The difference between $n$ and $h$ are the added outliers.
The lower bound of $h$ is set to be at 50\% because of the so called majority rule.
If the majority of a dataset contains added outliers, the estimator can't distinguish between inliers data structure and outlier data structure. \todo{cite source}

The MCD estimator is the most robust when 

\begin{equation}
    h = \lfloor \frac{n+p+1}{2} \rfloor \approx \frac{n}{2}
\end{equation}

wich corresponds to 0.5 at the population level. But the robusteness is been paid by efficency. 
If the dataset does not contain more than 25\% of outliers, than it is suitable to choose $h  0.75n$. \citep[p.217]{rousseeuw1999fast}
The newly calculated estimator is less robust, but more efficient then before. 

The resulting $\hat{\mu}_{\text{MCD}}$ is the mean vector of the remaining subset $h$. 
And t


The mean an covariance will be transformed accordingly.
This means that for any nonsingular p × p matrix A and
any p-dimensional column vector b it holds that

Therefore, transforming an h-subset with lowest determinant yields an h-subset XHA0 with lowest determinant among all
h-subsets of the transformed data set XA0, and its covariance matrix is transformed appropriately
Finally, we note that the robust mahalanobis distances are affine invariant, meaning they stay the same after transforming the data, which implies that the weighted
estimator is affine equivariant too.

To compute the MCD estimator one has to evaluate $\binom{n}{h}$ of all subsets size $h$.
Therefor \citet{rousseeuw1999fast} proposed an algorithn to increase calculation efficency. 
The algorithm of the concentration steps has the following steps \citep[p.214]{rousseeuw1999fast} 

\begin{enumerate}
    \item Initialize by drawing a random number $p+1$ subsamples $J$ and calculate the sample meand and the sample covariance matrix. If the resulting $det(S0)=0$ then add another random observation until $det(S0)>0$.
    \item If $det(S1)=/0$ compute the mahalanobis distances for all $n$
    \item To compute $H2$ sort the distances of (2) from smallest to biggest
    \item Select the $h$ smallest distances and compute the mean and coverance matrix
    \item Repeat until convergence
\end{enumerate}


\todo{add Algorithm FAST-MCD}
The general idea of FAST-MCD is not to compute all N over H combinations. That would be time consuming and not ressource efficient.
Instead it takes many initial choices of Subsets and are applying the concentrantion steps to each until convergence and keept the solution 
with the lowest determinant.
The algorithm steps are as follows:

\begin{enumerate}
    \item Determine the number observations $h$ by applying $h = \frac{n+p+1}{2}$
    \item carry out two concentration steps
    \item for the 10 results with lowest $det(S3)$: carry out C-steps until convergence
    \item report the mean and covariance matrix with the lowest $det(S)$
\end{enumerate}

As above mentioned, I will calculate just two concentration steps with all initial subsets.
After that I will select the ten results with the smallest determinants.
I will do this selective iteration, because the distinction between robust and nonrobust solutions aleady becomes visible after two or three steps.  
\todo{add:rousseeuw, Driessen (1999)p.216 pp, nested extensions and explain why the algorithm is like this}
If one can be sure, that the dataset contains less then 25\% of outliers, you can choose $h=0.75n$ wich is a good compromise between breakdown value and statistical efficiency.
If the dataset contains many observations (e.g. n > 600), then you can create

\textcolor{green}{
Briefly, statistical detection methods show efficient experi-
mental results when probabilistic distribution models are given
but suffer from high computational costs when applied in large
data sets and the curse of dimensionality. Therefore these methods
cannot be applied in both large and high dimensional data sets.
Another disadvantage of these techniques is that they are not
applicable to data sets where the distribution is unknown. \citep[p.5]{smiti2020critical}
}
\todo{Add: related work: }
Even recently reaseach has een done for new extension for MCD.
Because FAST-MCD begins by drawing random subsets, multiple runs of FastMCD on the same dataset may yield different results (a problem often mitigated by fixing the random seed in implementations). 
Furthermore, it requires drawing many initial subsets to ensure at least one is free of outliers, which can be computationally intensive. 
To circumvent both of these problems, the Deterministic MCD (DetMCD) algorithm was developed. \citep[p.621 pp.]{hubert2012deterministic}
DetMCD provides a deterministic approach to robust location and scatter. 
It uses the same iterative steps as FastMCD but avoids starting with random subsets, ensuring consistency. 
Unlike its predecessor, DetMCD is permutation invariant, meaning its result is independent of the order of observations in the dataset. 
Moreover, DetMCD typically runs even faster than FastMCD and is less sensitive to data contamination (outliers).


\subsection{Oulier detection methods based on machine learning}
\textcolor{green}{Definition Machine Learning}



\textcolor{red}{Over and Underfitting
Two common issues that affect a model's performance and generalization ability are overfitting and underfitting. 
These problems are major contributors to poor performance in machine learning models.
Bias and Variance: 
Bias is the error that happens when a machine learning model is too simple and doesn't learn enough details from the data. 
It's like assuming all birds can only be small and fly, so the model fails to recognize big birds like ostriches or penguins that can't fly and get biased with predictions.
High bias typically leads to underfitting, where the model performs poorly on both training and testing data because it fails to learn enough from the data.
These assumptions make the model easier to train but may prevent it from capturing the underlying complexities of the data.
In this case, the model doesn’t work well on either the training or testing data.
Variance: 
Error that happens when a machine learning model learns too much from the data, including random noise and outliers.
High variance typically leads to overfitting, where the model performs well on training data but poorly on testing data.
A high-variance model learns not only the patterns but also the noise in the training data, which leads to poor generalization on unseen data.
As a result, the model works great on training data but fails when tested on new data, because in other data can be other outliers and noise.
Overfitting models are like students who memorize answers instead of understanding the topic. 
They do well in practice tests (training) but struggle in real exams (testing)
The relationship between bias and variance is often referred to as the bias-variance tradeoff, which highlights the need for balance.
The goal is to find an optimal balance where both bias and variance are minimized, resulting in good generalization performance. 
(https://www.geeksforgeeks.org/machine-learning/underfitting-and-overfitting-in-machine-learning/)
Techniques to Reduce Underfitting: 
Increase model complexity
Increase the number of features, performing feature engineering.
Remove noise from the data
Increase the number of epochs or increase the duration of training to get better results.
Techniques to Reduce Overfitting
Improving the quality of training data reduces overfitting by focusing on meaningful patterns, mitigate the risk of fitting the noise or irrelevant features.
Increase the training data can improve the model's ability to generalize to unseen data and reduce the likelihood of overfitting.
Reduce model complexity.
Early stopping during the training phase (have an eye over the loss over the training period as soon as loss begins to increase stop training)
Use dropout for neural networks to tackle overfitting
Ridge Regularization and Lasso Regularization.
}

\subsubsection*{Cross-validation}

\textcolor{green}{text}
\textcolor{red}{Cross-validation
The problem is, that the test error depends on the test data  set.
The question is, how to estimate the test error using the training data only?
Cross-validation (CV) is often been used to hyperparamatic tuning.
CV randomly and repeatedly slits the data into training and test data.
Split the data set in $k$ equally large parts.
Part $i$ is the validation data set 
The model is trained using the remaining observations ans we compute the loss.
We repeat it for each validation data set $i = 1. ...  k$.
The final measure is 
Alternatives are: Elbow Method, Regularisation (?) 
}




\subsubsection*{k-Nearest-Neighbor}
The k-nearest neighbor (kNN) algorithm is a nonparametic classification algorithm. It is been known for its simplicity and accuracy.
It is a supervised learning algorithm, that means, that a labled traning dataset is been used to train the model and afterwards a class of unlabled data can be classified.
It is a "lazy learning" algorithm that means it does not achieve generalizations. The entire traningset is being used for traning.
When a new unlabled data is presented into the dataset, the k-NN algorithm determines in wich class the unlabled data should go by determining the wich classes the neighboring belong to.
Outlier detection is a binary classification problem. That means, that the considered neighbors for a new tuple, only can be inliers or outliers. 
If the 

There are two operations calculated, when new data is presented to the model:
\citep[p.1256 pp.]{9065747}

$K$ is the number of neighbors that would be considered for classifying a new unlabled tuple. 
Even though this classifier is simple, the value of ‘K’ plays an important role in classifying the unlabeled data.
$K$ is an hyperparameter, that means it is an external configuration variable that need be determined befor the process begins.
It is task of hyperparamatic tuning to find a optimal number of $k$. 
You have to consider the bias-variance-trade of when you choose the optimal $K$. 
The smaller $K$ is, the more likley it is that the model suffers from overfitting, leading to higher variance and lower bias.
The resulting decision boundaries will be complex and fragmented.
The lager $K$ ist, the more likley it is, that the model suffers from underfitting, leading to smaller variance and lagrer bias.
The resulting decision boundaries maybe missing local patterns.
It is advised to choose a odd-number $K$, otherwiseat at an even classification the model could fail.  

To determine the nearest neighbor, you need to calculate the distance between the new tuple and the neighbors. 
In classical k-NN, you use the Euclidean distance. 
By using the Euclidean-distance , k-NN is not affine equivariant.
Non-uniforly changes in the scale changes the distance landscap and thus leading to different set of nearest neighbors.
Resulting in different classifications. 

k-NN has many advantages. Its results are easy to understand and interpret.
The algorithm stays efficient with large traning sets eventhough it is invfluence by the curse of dimensional
There is no need for complex mathematical calculations, making it user-friendly.
Its is a non-paramatic model, that does not require any assumptions about the underlying distribution od the dataset.
That makes it flexible for complex datasets.

Some disadvantage of k-NN might be: the computational complexity. 
For a classification the model requires to store the entire dataset. Especially for large datasets, that could lead to slow prediction time. 
Moreover, storing the entire dataset is memory-intensive.
Because of the dependence of distance measurments, noisy data can influence the accurarcy negatively.
Chosing $K$ is vital for the models results. Through hyperparamatric tuning, one has to optimize the model. 

\begin{figure}[h] 
    \centering
    \includegraphics[width=1\linewidth]{kNN/k-NN.png}
    \caption{grapfical depiction of k-NN algorithm}
    \label{fig:my_example}
\end{figure}

\textcolor{red}{\citep{yang2023outlier} for further information}


\newpage

\subsubsection*{Contrastive Learning (Intercont)}

\todo{insert whats about and how will you write this passage}
\textcolor{green}{procedure}
explain what contrastive learing is.
1 whats a Neural network: whats are the differences between deep learning and statistical models?
    a General informations
        where do we use it 
        what are these tasks
        why do we use them
        Learning Paradigms
        Types of NN
        Feet forward vs feet backward networks
        advantages disadvantages 
        Differences between statistics and deep learning (k-NN defined as Machine Learning)

    b Details: 
        Arcitecture: Layers, defining, how many should you consider,
        Nodes, defining, which number to choose,
        Weights, definngm what happens to them
        Bias, defining, why do you need bias
        Activtion function, defining, hat type of af are there, differences
        Loss function, defining, differences, 
        Batch
        learning rate
        Unsupervised learning (its already in 2.0!)
        supervised learning
        reinforcement learning

2 Contrastive learning: A specially used Neural Network - specific
    a General informations  
        what is it
        where do we use it 
        why do we use it
        advantages and disadvantages
        loss functions 
    b Details   
        Use the specific details used in paper
        how is the 
\textcolor{red}{Neural Networks}
This section is about the newly proposed Intercont method of \citep{shenkar2022anomaly}. 
Included in the method, the authers used neural networks(NN).
I will begin by explaining the most common learning paradigms.
I will then describe what a NN is and briefly identify the use cases for it.
After that I will analyse the architecture and explore different types of networks including feedforwanrd and feedbackward networks.
Thereafter I will describe the training process and differentiate bewteen different types of learning methods.
Finally I will describe InterCont method.


A neural network is a model inspired by the structure and function of the human brain. 
It is suitable for a lot of tasks: for example classification, clustering, pattern recognition and predictive tasks.
The goal is to construct a generalizable model.
That means, that the model performs on unseen data nearly as good as on the training data \todo{check statement}
\todo{insert example papers maybe in the financial domain?}


There are different learning paradigms in machine learning. Nevertheless those depicted afterwards are somewhat of the extremes, because there a mixtures of all paradigms.
The most widley used paradigms is the supervised learning paradigm. 
In supervised Learning, the model ist tranined on labled data. The label defines the correct anwer for the model in advance.
The goal is for the model to learn a mapping function, so if a new unseen data is presented to the model, the model predicts the correct label.
It learns by comparing its predictions to the correct labels and minimizes the loss.
On the one hand, supervised learning leads to results with high accurarcy, reliable predictions and its performance is easy to measure.
On the other hand, it requires labled data, wich are highly cost and time intensive to generate.
Furthermore it could struggle with new unseen scenarios that are not in the training data.


In unsupervised learning, the training data is unlabled. It is only provided input data but no corresonding output lables or supervision.
Thre goal is to explore the data, find hidden patterns, and discover the underlying structure or distribution within the data on its own.
Its about drwaing inferences rather than making predictions with predefined answers.
For unsupervised learning there is no need for costly curated datasets, and the models can handle large data volumes.
It is perfect for finding hidden patterns, structures groupings and relationsships in the data.
But the models are not as accurate as supervised counterparts and its interpretability is more complex because there is no objective trith to measure against.
The models tend to be computational intensive espacially with large datasets. The results tend to be noisy.
\todo{insert example paper}


A hybrid approach is semi-supervised learning. A small amount of labled data combied with a large amount of unlabeled data for trainin is used to gain 
to understand the overall structure and improve the models accurarcy. Therefor the initial tranining is been done with labled data and afterwards its predictions 
will be used on the unlabled data as "pseudo-lables".
Semi-Supervised learning tries to mitigate the disadvantages of the other ones. 
The labeling cost are reduce, because the labeled datasets are smaller than in supervised learning. It can use the vast amount of unlabled data available in many real world domains.
Its generalizations are improved By leveraging the large unlabeled dataset, the model gains a better understanding of the overall data structure,
which can lead to better performance on unseen data than purely supervised models trained on limited labels. 
The disadvantage lays in it initial small labled dataset. If its first predictions are  inaccurate, these errors can be amplified for the unlabled data.
Moreover it requires careful selection of tuning algorithms, as different semi-supervised methods have varying effectiveness based on the data. 


\todo{consider: reinforcement learning}
further information for every paradigm, see \citep{emmert2022taxonomy}

Self-supervised learning can be used to mittigate the problem of costly data labeling.
In cases, e.g. Computer Vision, where a potentially a high number of classes can emerge, it is more suitable to design a model wich can fulfill this task alone. 
Introducing metric learning, where the goal is to generate a numerical representation of the input, as a image, and putting it into an embedding space resulting in a vector.
This initial vector is called anchor. New images are added and the network learns an embedding space, where simmiliar images are put cloder together and dissimiliar 
images are put further away. When the network is presented with a slightly different version of the anchor image, the network is capable of recognizing the most important 
aspects of this image, because it puts the new image near the anchor in the embedding space.  
Those networks reduce complexity by reducing the dimensionality of the input.
If a new datapoint is introduced, the model embedds the data and and classifies the data accourdingly to the localtion in the embedding space. 


The arcitecture of a neural network consists of layers. Those layers consist of nodes, wich are mathematical functions.
The first layer is the input layer. It receives the input data. \colorbox{red}{Each node in the input layer corrensponds to a feature in the data.}
After performing the first computation through the node, the result is pushed to the next layer, called hidden layer.
The network can be made up of aribrary number of hidden layers. 
Between layers there are the weights, biases and activation functions. 
The weight determines how important the connection between these two nodes are. The more imporant it is, the higher the weight. 
The bias is a threshold value that determines if the node should pass the result to the next layer.
If the threshold is met, the node is passing the result to the next layer.


The activation function is used to bring non-linearity into the model.
\todo{insert Leaky RELU, its slope coefficient and Tanh activation functions}  
Without an activation function, the model just predict linear results, because a linear function of a linear function is a linear function.
The model passes the input and multiplies it with the weight and adds a bias. 
The result is def into the activation function. If the function is met, the activation function passes the result onto another layer.


The last layer is called an output layer. 
The resulting output is compared to the known actual result, through a loss function.
We need a performance measure to quantify the goodness of the learner, hence a loss function. 
It is a function of the true targets and the learned predictions.
The goal is to minimize the the loss, so that the model's predictions are more accurate.
But a perfectly fitting model can lead to the problem of overfittig.
There are plentyful of different options to choose from.
The most common loss functions are Mean Squared Error (MSE), Mean Absolute Error (MAE) and Huber Loss.
The Mean Squared Error is defined as: 
\begin{equation}
    \text{MSE} = \frac{1}{n} \sum_{i=1}^{n} (Y_i - \hat{Y}_i)^2
\end{equation}
The sum is squared because the difference between prediction and true target is defined as non-negative.
The MSE penalizes large prediction errors more heavily.
It is equivalent to the OLS objective function and is consistent with the normality assumption.
The Mean Absolute Error is defined as: 
\begin{equation}
    \text{MAE} = \frac{1}{n} \sum_{i=1}^{n} |Y_i - \hat{Y}_i|
\end{equation}
It uses the median instead of the mean for a more robust loss.
The Huber loss is a combination of MSE and MAE to robustify MSE and id defined as:
\begin{equation}
    L_{\delta}(y, \hat{y}) =
    \begin{cases}
    \frac{1}{2}(y - \hat{y})^2 & \text{for } |y - \hat{y}| \le \delta \\
    \delta |y - \hat{y}| - \frac{1}{2}\delta^2 & \text{otherwise}
    \end{cases}
\end{equation}
\todo{insert contrastive loss function of paper}
\todo{insert: most common loss functions and paper examples}


After calculating the loss, the model uses backpropagation to send the result backwards thought the layers.  
By using the chain rule, the model calculates how much each weight and bias contributes to the loss, called gradient.
An optimizer is been used to adjust the bias and weights. The goal is to minimize the loss function.
The loss is calculated again, fulfilling an epoch. 
\todo{consider: example of chain rule?}
\todo{example of optimizer}


Models that use more than two hidden layers are often referred to as "deep networks".
\todo{insert: deep learning, differences, advantages and disadvantages}


Models that use this iteration process are called feedforward neural neutworks. 
The calculation are pushed in just one direction.
A different type of neural networks are the so called feedback neural networks. 
\todo{insert: feedbackward network, use cases disadvantages and advantages}
\todo{insert Diagram}


Once the augmented instances are encoded and projected into the embedding space, the contrastive learning objective is applied. 
The objective is to maximize the agreement between positive pairs (instances from the same sample) and minimize the agreement between negative pairs (instances from different samples).
This encourages the model to pull similar instances closer together while pushing dissimilar instances apart. 
The similarity between instances is typically measured using a distance metric, such as Euclidean distance or cosine similarity. 
The model is trained to minimize the distance between positive pairs and maximize the distance between negative pairs 
in the embedding space. (https://encord.com/blog/guide-to-contrastive-learning/)
\todo{decide: keep it, where should it go?}


\textcolor{red}{Intercont Method}
In the paper of \citeauthor{shenkar2022anomaly} they propose a novel outlier detection method, based on a self-supervised contrastive learning alorithm.
First I will describe how the arcitecture is designed.


The goal is to produce an score, that is low for similar data pairs and high for dissimilar pairs.
The input data of the InterCont Method is supposed to be of tabular data.
The training set contains n in-class samples x. All of them are vectors of d dimensions.
If the sample is not from the underlying distribution it gets high score, thus is more likely a outlier. 
First a subsample is taken of k consecutive variables and m = d + 1 - k pairs %(a,b)% are generated.
Whereas k determines the number of variables that are considered and there cant exceed d dimensions.
a is the vector including all variables k and b is the vector of the rest d - k. 
The pairs are complementing each other. 
The method learns mapping yb using two networks F,G. Each network facilitates a or b.
The mutual information is been maximized by using the noise contrastive estimation framework of \citep{oord2018representation}.
In this framwork b is used as a query and the complementary a vector is used as a positive sample. Otherwise it is a negative sample.
Contrestive learning minimizes the distance bewteen query and positive sample and maximizes the distance bewteen query and negative sample.
Thats calles a masking effect, because 

The model consists two fully connected neural networks F and G. F has two hidden layers. The first hidden layer has 200 nodes and the secound has 400.
The second Layer is equally build with two layers, but the input data is smaller, so the number of nodes differ.
In the first hidden layer are 50 nodes and in the second are 100 nodes. As a gernalisation the authers specified the number of nodes as dependent on the embedding size u.
They took the LeakyRELU activation function except of the first hidden layer of F. There they use a Tanh activation. 
The LeakyRELU is different from the describtion above. It introduces a small non-zero slope coeficient for negative inputs. 
The authers use a slope coefficient of 0.2.
That prevents neurons from dying, so they still contribute. Moreover it ensures that gradients can propagate backwards through the network eben for negativ activations, 
reducing the risk of vanishing gradients.


The result are applied to a Batch Normalization so it re-centers and rescales the data for faster, more stable and less sensitive to hoe you initialize weights
\todo{insert details how its done and why}



\todo{insert double normalization}

The encoder network takes the augmented instances as input and maps them to a latent representation space,
where meaningful features and similarities are captured. 
A latent space is a abstract, compressed and low-dimensional vector space of the input data, where essential features and underlying patterns are encoded.
The goal is to filter out noise and redundant information. 
It perserves semantic relationsships. Data points that are similar in the original, will be mapped to points that are close together. 
Points that are dissimilar will be far apart.
The latent space is produced by representation learning. 
Autoencoders are the most common way to learn a latent space. They consists of two main parts:
The encoder takes the high dimensional input and maps it to a much smaller vector in the latent space. This enforces the necessary compression.
The middle layer of the autoencoder is the bottleneck, wich contains the ompressed, low dimensional representation. 
The Decoder takes the compressed vector and attemps to reconstruct the original high dimensional input. 
The autoencoder is trained to minimize the difference between the original input and the reconstructed output. By forcing the information through the bottleneck, the 
enoder learns only the most essential features required for accurate reconstruction.

\textcolor{green}{Encoder Network}
To maximize the mutual information gain, the Intercont methods uses an encoder network based on noise contrastive estimation. \citep{oord2018representation}
\todo{do I need this?}



\textcolor{green}{In Paper}
The contemporaneous NeuTraL AD work by Qiu et al. (2021) employs per-sample contrastive loss
for identifying anomalies in tabular data, similar to our work. However, there are crucial differences:
(1) NeuTraL AD learns specific masks, while we apply the entire set of the masks specified by a
window size k. (2) The role of NeuTraL AD masks is to mask-out parts that are irrelevant for specific
classes. In our case, we perform a two sided matching that identifies the masked part form the original.
(3) NeuTraL AD learns a single feature extractor (“encoder”) for both the original and transformed
data. In our case, the two sides of the contrastive loss are of very different dimensions $d - k$ and $k$
and we employ two different encoders. \cite[p.3]{shenkar2022anomaly}



\textcolor{red}{Projection Network 
A projection network is employed to refine the learned representations further. 
The projection network takes the output of the encoder network and projects it onto a lower-dimensional space, 
often referred to as the projection or embedding space. 
This additional projection step helps enhance the discriminative power of the learned representations. 
By mapping the representations to a lower-dimensional space, the projection network reduces the complexity and redundancy 
in the data, facilitating better separation between similar and dissimilar instances.(https://encord.com/blog/guide-to-contrastive-learning/)
}






\textcolor{green}{
Example   
}




Contrastive Learning
latent space
embedding size / embedding space 
temperatur constant of the loss 
mutual information
L2 Norm 
cross entropy loss Adam optimzier 
LeakyRELU activation 
bagging efferct
slope coefficient 
tanh activation 
Batch normalization 
The Ordinary Least Squared was being critized because of its lack of robusteness. Just one outlier could have a large negative effect.

Mahalanobis disctance cant detect multiple outliers because of the masking effect, by wich multiple outliers dont have necesssarly a large mahananobis disctance (R99 S.212)
Main idea of MCD find h observations out of n whose classical covariance matrix has the lowest determinant. The MCD estimate is then rhe average of these h points 
and the MCD estimate of scatter is their covariance matrix. 

Procedure: 1. Creating initial subset. 
2 Possibillities: draw a random h subset H1, or draw a random (p+1)-subset auf J, and then compute 
technical terms:

Determinant - Sie ist das Produkt der Hauptdiagonalen minus dem Produkt der Nebendiagonalen. Das ist stimmt bis zu einer 3x3 Matrix, darüberhinaus werden andere Methoden verwendet (Laplace'scher Entwicklungssatz)
Es ist eine Eigenschaft die Auskunft über die Volumenverzerrung und Inventierbarkeit einer Matrix gibt. 
Inventierbar ist eine Matrix, wenn det (A) =/ 0. 
Geometrische Bedeutung:  Der Betrag der Determinante ist gleich dem Faktor, um den sich das Volumen (oder Fläche)  eines Einheitswürfels nach der Transformation vergrößert oder verkleinert.
Das Vorzeichen der det(A) gibt Auskunft über die Orientierung. det (A) > 0 bleibt Orientierung erhalten (zb Drehung oder Streckung)
Wenn det(A) < 0 wird die Orientierung umgekehrt (z.B. eine Spiegelung)

asymptotisch normalverteilt 

\newpage 
\newpage

Minimum Coavariance Determinant is asymptotically normal and robust. But it is hard to compute.


Therefor the authors constructed an algorithm, thats much faster than the first algorithm.
MCD Alternative: for dataset with very few observations/ excessive noise use Minimum Regularized Covariance Determinant
Insert: the curse of dimensionality

For the end: how can I be sure, that my results are legit?

for dataset analysis: multicolinearityt?



What is the solution?
The Minimum Covariance Determinant, first published in 1985 by , is a robust estimator of the covariance matrix in multidimensional dataset.
Given a dataset of n observations and a predetermined subset sie h, the MCD searches for the h observations whose empirical covariance matrix has the smalltest possible determinant.
A determinant is a mathematical property of a matrix.

How does it work?


What are the advantages?

What are the disadvantages?