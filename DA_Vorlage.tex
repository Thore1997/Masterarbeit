\documentclass[a4paper, 	% Seitenformat
		12pt, 								% Schriftgr��e
		bibliography=totoc, 	% Literaturverzeichnis in das Inhaltsverzeichnis
		index=totoc, 	% Index in das Inhaltsverzeichnis
		abstract=true, 	% mit Abstrakt
		headsepline, 	% Trennlinie f�r die Kopfzeile
		%footnosepline,	% Trennlinie f�r die Fusszeile
		]{scrartcl}

\usepackage[T1]{fontenc}
\usepackage[utf8]{inputenc}
\usepackage{lmodern}
\usepackage[ngerman]{babel}
%\usepackage{ulem}
% Bilder
%\usepackage{pst-all}		% Zeichnungen in Latex (kein pdflatex)
\usepackage{subfig}	  	% Unterteilungen von Abbildungen
\usepackage{float}		  % Gleitobjekte
\usepackage{graphicx}		% Einbinden von Graphiken aus Dateien
\usepackage{rotating}   % Die folgenden Packages sind meist hilfreich f�r das Erstellen von Arbeiten
\usepackage{mathrsfs}
\usepackage{eurosym,bm,amsmath,amssymb,verbatim}
\usepackage{listings}
\usepackage[T1]{fontenc}
\usepackage{setspace}
\usepackage [round,authoryear] {natbib} % Notwendiges Package f�r den verwendeten Literaturstil
%\usepackage{natbib}

\usepackage[colorinlistoftodos,prependcaption,textsize=tiny,textwidth=2cm]{todonotes}
%------------------------------------------------------------------------

\parindent0em 			% kein Einzug nach einer Leerzeile
\topmargin-20.4mm   % F�r die R�nder
\textheight245mm
\textwidth157mm \oddsidemargin11mm \evensidemargin11mm
\hyphenpenalty=100
\exhyphenpenalty=100

%%%%%%%%%%%%%%%%%%%%%%%%%%%%%%%%%%%%%%%%%%%%%%%%%%%%%%%%%%%%%%%%%%%%%%%
%%%%%%%%%%%%%%%%%%%%%%%%%%%%%%%%%%%%%%%%%%%%%%%%%%%%%%%%%%%%%%%%%%%%%%%
%%%%%%%%%%%%%%%%%%%%%%%%%%%%%%%%%%%%%%%%%%%%%%%%%%%%%%%%%%%%%%%%%%%%%%%

\begin{document}					% Start des Dokuments
\begin{titlepage}

\hspace{12cm}\scalebox{0.3}{\includegraphics{Unilogo.jpg}}

\vspace*{1cm}
\large\textbf{Disposition: }\\[0.5cm]
\large\textbf{From Statistics to Machine Learning: \\ A Comparative Study of Outlier Detection Methods \\ in Multidimensional Data}\\[0.5cm]

Studiengang: Betriebswirtschaftslehre\\

Lehrstuhl für Statistik\\[1.5cm]
%
%
\noindent\begin{tabular}[h]{@{}ll}
%\begin{tabular}{l l}
Eingereicht bei: &Prof. Dr. Yarema Okhrin \\
\\
Betreuer: &Herr Philipp Haid 			 \\
\\
Vorgelegt von:   &Thore Johannsen         \\
\\
Adresse: &Salomon-Idler-Straße 4         \\[0.2cm]
                 &Augsburg                      \\
\\
Matrikel-Nr.:    &2231165                 \\
\\
Email:           &thore.johannsen@outlook.com     \\
\\
\end{tabular}
\vfill
Augsburg, im September 2010
\end{titlepage}
\newpage
\pagenumbering{Roman}


%\tableofcontents		    % Inhaltsverzeichnis
%\newpage
%\listoffigures			    % Abbildungsverzeichnis
%\listoftables				  % Tabellenverzeichnis

\clearpage
\pagenumbering{arabic}
\onehalfspacing 			     % F�r richtigen Zeilenabstand

%%%%%%%%%%%%%%%%%%%%%%%%%%%%%%%%%%%%%%%%%%%%%%%%%%%%%%%%%%%%%%%%%%%%%%%
%%%%%%%%%%%%%%%%%%%%%%%%%%%%%%%%%%%%%%%%%%%%%%%%%%%%%%%%%%%%%%%%%%%%%%%
%%%%%%%%%%%%%%%%%%%%%%%%%%%%%%%%%%%%%%%%%%%%%%%%%%%%%%%%%%%%%%%%%%%%%%%

%\input{Kapitel/kapitel1}  % Zur besseren Strukturierung empfiehlt es sich das Dokument aufzuteilen.
%\newpage
%\input{Kapitel/kapitel2}
%\newpage
%\input{Kapitel/kapitel3}
%\newpage
%\input{Kapitel/kapitel4}
%\newpage
%\input{Kapitel/kapitel5}
%\vspace{5em}

%%%%%%%%%%%%%%%%%%%%%%%%%%%%%%%%%%%%%%%%%%%%%%%%%%%%%%%%%%%%%%%%%%%%%%%
%%%%%%%%%%%%%%%%%%%%%%%%%%%%%%%%%%%%%%%%%%%%%%%%%%%%%%%%%%%%%%%%%%%%%%%
%%%%%%%%%%%%%%%%%%%%%%%%%%%%%%%%%%%%%%%%%%%%%%%%%%%%%%%%%%%%%%%%%%%%%%%

This thesis undertakes a comparative performance analysis of three outlier detection methods: the FAST-MCD algorithm \citep{rousseeuw1999fast}, the k-Nearest Neighbors (kNN) approach \citep{ramaswamy2000efficient}, and the contrastive learning–based method proposed by \citet{shenkar2022anomaly}.
Outlier detection constitutes an essential component across scientific research and industrial applications. \\
The presence of anomalous data points can compromise statistical inference. 
Specifically, outliers possess the capacity to inflate estimates of variability (e.g. variance and standard deviation), consequently widening confidence intervals. 
This effect increases the probability of committing a Type II error (failure to reject a false null hypothesis). 
Conversely, in certain domains, the outliers themselves represent the primary interest. 
Prominent examples include the application in financial fraud detection \citep{ramaswamy2000efficient} and in medical diagnostics for identifying pathological conditions, such as cervical cancer \citep{ijaz2020data}. \\
The primary objective of this investigation is to systematically evaluate the relative efficacy of the three selected algorithms in identifying outliers within a designated dataset.
Following the formal definition and characterization of outliers, the study will detail the underlying mechanisms of the three analyzed methods. 
This paper will begin with the robust statistical method of FAST-MCD. 
Subsequently, the analysis will proceed to discuss the distance-based method of kNN. 
Finally, the Intercont approach, a novel machine learning method will be presented.
Different performance metrics will be discussed to supply a optimal view of the detection performance. 
The empirical investigation will utilize financial datasets, drawing among other things primary research literature.
The study is guided by the following formal hypotheses regarding detection efficacy:
The kNN algorithm is hypothesized to yield the highest anomaly detection performance among the three candidates.
The contrastive learning–based Intercont algorithm is anticipated to outperform the FAST-MCD method but is expected to demonstrate lower efficacy than the kNN algorithm.

\newpage
\section {Einleitung}
\newpage
\section {Definition Outlier Analysis}

Outlier Analysis is an important task for research and real world applications like spam filters.
Suspicious observations should be marked as Outliers and inliners should not be marked as outliers. 
Ouliers, also called anomalies \citep{chandola2009anomaly}, are observations, that deviate extremely from the defined norm.
They are the product of a different mechanisms, that aren't included into the research question. \citep{hawkins1980identification}
On the one hand, depending on the context they can be researchers target and provide useful information.
For instance outliers develope by measuring recording errors, misreporting or sample errors.
E.g. in financial domains, outliers are been used to identify criminal activity like creditcard fraud. 
The usage of the stolen credit card by the criminal should be very different of the usage of the rightful credit card holder. \citep{rezapour2019anomaly}
On the other hand, if they are not the target, they have to be identified and carefully treated. \citep{smiti2020critical}
You can treat them by removing, transforming, winzoriesing or using robust alternatives.
Every method has its advantages and disadvantages, but it would be out of scope of this thesis to go deeper into it. 


Different from outliers is noisy data.
Noisy data has no useful information and hinders the analysis. 
It need to be identified and corrected because it makes computation difficult or impossible to do.
Noisy data contains incorrect data types, missing values or incorrect data. \citep{smiti2020critical}
There are 3 different types of outliers: 


\begin{enumerate}
\item Global outliers, sometimes named point anomalies, are the simplest type of anomaly. 
When a obervation with respect to rest of the its distribution has extreme values, you could consider it a outlier.
E.g. a Person spends 100€ a week and someday the person spends 1000€. that would be possibly be a global outlier, because it is 
a extreme value out of the normal distribution. 
\item Contextual outliers, sometimes named conditional outliers, are outliers that are outliers bescause of their specific contexts.
A data instance is defined by it behavioral and contextual attributes. 
Contextual attributes define contextual atrributes and behavioral attributes define non-contextual attributes.
Using the credit card fraud example, behavioral atrributes are the amount of paymets done in a certain period.
Contextual attribute is the time dimension of this longitudinal data. 
If a person spends 100€ in a week, accept for christmas holiday. There the Person spends 1000€.
the contextual outlier would be, if a transaction of 1000€ emerges in mid july.
1000€ is not an extreme charakteristic, because it will be spend in a different time.
But it is a unusual period of time. 
\item Collective outliers are observations that are by itselfe not unsual but their accurance together as a collection. 
Using the example as before, a single use of a credit card for a normal payment is not a outlier, but when a person pays 50 times
a small payment, that would be considered a collective outlier.
\end{enumerate}


All 3 are closely related. A global outlier or a collective outlier can also be a contextualized and treated like a 
contextual problem. \citep{chandola2009anomaly}


\newpage


What is novelty detection?
(C09 S.3)	wich aims at detectiting previously unobserved patterns in the data, for example, a new topic of discussion in a news group
			The difference between novel patterns and anomalies is that the novel patterns are typically incorporated 
			into the normal model after being detected. 
			Solutions for these related problems are often used for anomaly detection and novelty detection

What is outlier explanation?
(P22 s.978)	In fields where ouliers do have meaningfull information, it is necessary to generate an importance level 
			of outliers bescause users have make decisions based of prioritization
(P22 S. 983f)e.g. Fraud detection: Credit Card fraud detection deals with millions of transactions. 
			To reduce false-positive detection ad the costs that come with it, humans analysts are monitoring 
			all suspicious transactions.
			If those suspicious transactions would be prioritized, it could lead to a more efficient result. 

			
\newpage 


			\clearpage
\bibliographystyle{autorjahrdidiDE}  % Dies ist der verwendete Stil f�r Zitation und Literaturverzeichnis																	 % Die Datei autorjahrdidiDE.bst muss in das aktuelle Arbeitsverzeichnis kopiert werden
\bibliography{Literatur}	           % Literaturverzeichnis liegt in der Datei Literatur
\end{document}											 % Ende des Dokuments
