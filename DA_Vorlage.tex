\documentclass[a4paper, 	% Seitenformat
		12pt, 								% Schriftgr��e
		bibliography=totoc, 	% Literaturverzeichnis in das Inhaltsverzeichnis
		index=totoc, 	% Index in das Inhaltsverzeichnis
		abstracton, 	% mit Abstrakt
		headsepline, 	% Trennlinie f�r die Kopfzeile
		%footnosepline,	% Trennlinie f�r die Fusszeile
		]{scrartcl}

\usepackage[T1]{fontenc}
\usepackage[utf8]{inputenc}
\usepackage{lmodern}
\usepackage[ngerman]{babel}
%\usepackage{ulem}
% Bilder
%\usepackage{pst-all}		% Zeichnungen in Latex (kein pdflatex)
\usepackage{subfig}	  	% Unterteilungen von Abbildungen
\usepackage{float}		  % Gleitobjekte
\usepackage{graphicx}		% Einbinden von Graphiken aus Dateien
\usepackage{rotating}   % Die folgenden Packages sind meist hilfreich f�r das Erstellen von Arbeiten
\usepackage{mathrsfs}
\usepackage{eurosym,bm,amsmath,amssymb,verbatim}
\usepackage{listings}
\usepackage[T1]{fontenc}
\usepackage{setspace}
\usepackage [round,authoryear] {natbib} % Notwendiges Package f�r den verwendeten Literaturstil
%\usepackage{natbib}





%------------------------------------------------------------------------

\parindent0em 			% kein Einzug nach einer Leerzeile
\topmargin-20.4mm   % F�r die R�nder
\textheight245mm

\textwidth157mm \oddsidemargin11mm \evensidemargin11mm

%%%%%%%%%%%%%%%%%%%%%%%%%%%%%%%%%%%%%%%%%%%%%%%%%%%%%%%%%%%%%%%%%%%%%%%
%%%%%%%%%%%%%%%%%%%%%%%%%%%%%%%%%%%%%%%%%%%%%%%%%%%%%%%%%%%%%%%%%%%%%%%
%%%%%%%%%%%%%%%%%%%%%%%%%%%%%%%%%%%%%%%%%%%%%%%%%%%%%%%%%%%%%%%%%%%%%%%

\begin{document}					% Start des Dokuments
\begin{titlepage}

\hspace{12cm}\scalebox{0.3}{\includegraphics{Unilogo.jpg}}

\vspace*{1cm}
\large\textbf{Disputation: }\\[0.5cm]
\large\textbf{From Statistics to Representation Learning: \\ A Comparative Study on Outlier Detection Methods}\\[0.5cm]

Studiengang: Betriebswirtschaftslehre\\

Lehrstuhl für Statistik\\[1.5cm]
%
%
\noindent\begin{tabular}[h]{@{}ll}
%\begin{tabular}{l l}
Eingereicht bei: &Prof. Dr. Yarema Okhrin \\
\\
Betreuer: &Herr Philipp Haid 			 \\
\\
Vorgelegt von:   &Thore Johannsen         \\
\\
Adresse: &Salomon-Idler-Straße 4         \\[0.2cm]
                 &Augsburg                      \\
\\
Matrikel-Nr.:    &2231165                 \\
\\
Email:           &thore.johannsen@outlook.com     \\
\\
\end{tabular}

\vfill
Augsburg, im September 2010

\end{titlepage}


\newpage
\pagenumbering{Roman}


%\tableofcontents		    % Inhaltsverzeichnis
%\newpage
%\listoffigures			    % Abbildungsverzeichnis
%\listoftables				  % Tabellenverzeichnis

\clearpage
\pagenumbering{arabic}
\onehalfspacing 			     % F�r richtigen Zeilenabstand

%%%%%%%%%%%%%%%%%%%%%%%%%%%%%%%%%%%%%%%%%%%%%%%%%%%%%%%%%%%%%%%%%%%%%%%
%%%%%%%%%%%%%%%%%%%%%%%%%%%%%%%%%%%%%%%%%%%%%%%%%%%%%%%%%%%%%%%%%%%%%%%
%%%%%%%%%%%%%%%%%%%%%%%%%%%%%%%%%%%%%%%%%%%%%%%%%%%%%%%%%%%%%%%%%%%%%%%

%\input{Kapitel/kapitel1}  % Zur besseren Strukturierung empfiehlt es sich das Dokument aufzuteilen.
%\newpage
%\input{Kapitel/kapitel2}
%\newpage
%\input{Kapitel/kapitel3}
%\newpage
%\input{Kapitel/kapitel4}
%\newpage
%\input{Kapitel/kapitel5}
%\vspace{5em}

%%%%%%%%%%%%%%%%%%%%%%%%%%%%%%%%%%%%%%%%%%%%%%%%%%%%%%%%%%%%%%%%%%%%%%%
%%%%%%%%%%%%%%%%%%%%%%%%%%%%%%%%%%%%%%%%%%%%%%%%%%%%%%%%%%%%%%%%%%%%%%%
%%%%%%%%%%%%%%%%%%%%%%%%%%%%%%%%%%%%%%%%%%%%%%%%%%%%%%%%%%%%%%%%%%%%%%%

\section*{Thesis Proposal}

This thesis presents a comparative analysis of the FAST-MCD, k-Nearest Neighbors (kNN), and a contrastive learning–based approach proposed by \citet{shenkar2022anomaly}.
Outlier detection plays a crucial role in both scientific research and industrial applications. 
Outliers can substantially influence statistical measures such as the variance and standard deviation of variables. 
Consequently, confidence intervals become wider, which may lead to a failure to reject the null hypothesis even when a true effect exists.
In some contexts, however, outliers represent the primary objects of interest. 
For example, banks employ outlier detection techniques to differentiate between fraudulent and legitimate credit card usage \citep{ramaswamy2000efficient}. 
Similarly, in medicine, outlier detection has been applied to identify diseases such as cervical cancer \citep{ijaz2020data}.
The objective of this thesis is to evaluate which of the three selected methods performs most effectively in detecting outliers. 
It is hypothesized that the kNN algorithm will achieve the highest detection performance. 
Furthermore, it is expected that the contrastive learning–based algorithm will outperform the FAST-MCD method but remain less effective than kNN.
The Minimum Covariance Determinant (MCD) algorithm will be described first, as it constitutes the foundation of the FAST-MCD approach. 
To this end, the Mahalanobis distance \citep{mahalanobis2018generalized}, which is integral to the MCD algorithm \citep{rousseeuw1984least}, will be explained. 
Subsequently, the extension of MCD to FAST-MCD will be discussed, as this variant will be implemented in the empirical part of this thesis. 
FAST-MCD offers computational advantages by being significantly faster and more suitable for high-dimensional datasets compared to the original MCD algorithm.
Next, the k-Nearest Neighbor (kNN) algorithm \citep{ramaswamy2000efficient} will be introduced. 
This distance-based method is straightforward to implement and produces results that are easy to interpret. 
However, kNN is less suitable for large datasets because it requires comparison of each new observation with all training samples, resulting in slow prediction times.
The third method examined in this thesis is the Internal Contrastive Learning (InterCont) approach proposed by \citet{shenkar2022anomaly}. 
This recent method employs contrastive learning to train a model to differentiate between similar and dissimilar pairs of data samples. 
A notable advantage of this approach is that it does not rely on prior assumptions or information about the underlying data structure.



\clearpage

\bibliographystyle{autorjahrdidiDE}  % Dies ist der verwendete Stil f�r Zitation und Literaturverzeichnis
																		 % Die Datei autorjahrdidiDE.bst muss in das aktuelle Arbeitsverzeichnis kopiert werden
\bibliography{Literatur}	           % Literaturverzeichnis liegt in der Datei Literatur

\end{document}											 % Ende des Dokuments
