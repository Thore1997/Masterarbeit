\documentclass[a4paper, 	% Seitenformat
		12pt, 								% Schriftgr��e
		bibliography=totoc, 	% Literaturverzeichnis in das Inhaltsverzeichnis
		index=totoc, 	% Index in das Inhaltsverzeichnis
		abstract=true, 	% mit Abstrakt
		headsepline, 	% Trennlinie f�r die Kopfzeile
		%footnosepline,	% Trennlinie f�r die Fusszeile
		]{scrartcl}

\usepackage[T1]{fontenc}
\usepackage[utf8]{inputenc}
\usepackage{lmodern}
\usepackage[ngerman]{babel}
%\usepackage{ulem}
% Bilder
%\usepackage{pst-all}		% Zeichnungen in Latex (kein pdflatex)
\usepackage{subfig}	  	% Unterteilungen von Abbildungen
\usepackage{float}		  % Gleitobjekte
\usepackage{graphicx}		% Einbinden von Graphiken aus Dateien
\usepackage{rotating}   % Die folgenden Packages sind meist hilfreich f�r das Erstellen von Arbeiten
\usepackage{mathrsfs}
\usepackage{eurosym,bm,amsmath,amssymb,verbatim}
\usepackage{listings}
\usepackage[T1]{fontenc}
\usepackage{setspace}
\usepackage [round,authoryear] {natbib} % Notwendiges Package f�r den verwendeten Literaturstil
%\usepackage{natbib}

\usepackage[colorinlistoftodos,prependcaption,textsize=tiny,textwidth=2cm]{todonotes}
%------------------------------------------------------------------------

\parindent0em 			% kein Einzug nach einer Leerzeile
\topmargin-20.4mm   % F�r die R�nder
\textheight245mm
\textwidth157mm \oddsidemargin11mm \evensidemargin11mm
\hyphenpenalty=100
\exhyphenpenalty=100

%%%%%%%%%%%%%%%%%%%%%%%%%%%%%%%%%%%%%%%%%%%%%%%%%%%%%%%%%%%%%%%%%%%%%%%
%%%%%%%%%%%%%%%%%%%%%%%%%%%%%%%%%%%%%%%%%%%%%%%%%%%%%%%%%%%%%%%%%%%%%%%
%%%%%%%%%%%%%%%%%%%%%%%%%%%%%%%%%%%%%%%%%%%%%%%%%%%%%%%%%%%%%%%%%%%%%%%

\begin{document}					% Start des Dokuments
\begin{titlepage}

\hspace{12cm}\scalebox{0.3}{\includegraphics{Unilogo.jpg}}

\vspace*{1cm}
\large\textbf{Disposition: }\\[0.5cm]
\large\textbf{From Statistics to Machine Learning: \\ A Comparative Study of Outlier Detection Methods \\ in Multidimensional Data}\\[0.5cm]

Studiengang: Betriebswirtschaftslehre\\

Lehrstuhl für Statistik\\[1.5cm]
%
%
\noindent\begin{tabular}[h]{@{}ll}
%\begin{tabular}{l l}
Eingereicht bei: &Prof. Dr. Yarema Okhrin \\
\\
Betreuer: &Herr Philipp Haid 			 \\
\\
Vorgelegt von:   &Thore Johannsen         \\
\\
Adresse: &Salomon-Idler-Straße 4         \\[0.2cm]
                 &Augsburg                      \\
\\
Matrikel-Nr.:    &2231165                 \\
\\
Email:           &thore.johannsen@outlook.com     \\
\\
\end{tabular}
\vfill
Augsburg, im September 2010
\end{titlepage}
\newpage
\pagenumbering{Roman}


%\tableofcontents		    % Inhaltsverzeichnis
%\newpage
%\listoffigures			    % Abbildungsverzeichnis
%\listoftables				  % Tabellenverzeichnis

\clearpage
\pagenumbering{arabic}
\onehalfspacing 			     % F�r richtigen Zeilenabstand

%%%%%%%%%%%%%%%%%%%%%%%%%%%%%%%%%%%%%%%%%%%%%%%%%%%%%%%%%%%%%%%%%%%%%%%
%%%%%%%%%%%%%%%%%%%%%%%%%%%%%%%%%%%%%%%%%%%%%%%%%%%%%%%%%%%%%%%%%%%%%%%
%%%%%%%%%%%%%%%%%%%%%%%%%%%%%%%%%%%%%%%%%%%%%%%%%%%%%%%%%%%%%%%%%%%%%%%

%\input{Kapitel/kapitel1}  % Zur besseren Strukturierung empfiehlt es sich das Dokument aufzuteilen.
%\newpage
%\input{Kapitel/kapitel2}
%\newpage
%\input{Kapitel/kapitel3}
%\newpage
%\input{Kapitel/kapitel4}
%\newpage
%\input{Kapitel/kapitel5}
%\vspace{5em}

%%%%%%%%%%%%%%%%%%%%%%%%%%%%%%%%%%%%%%%%%%%%%%%%%%%%%%%%%%%%%%%%%%%%%%%
%%%%%%%%%%%%%%%%%%%%%%%%%%%%%%%%%%%%%%%%%%%%%%%%%%%%%%%%%%%%%%%%%%%%%%%
%%%%%%%%%%%%%%%%%%%%%%%%%%%%%%%%%%%%%%%%%%%%%%%%%%%%%%%%%%%%%%%%%%%%%%%

This thesis undertakes a comparative performance analysis of three outlier detection methods: the FAST-MCD algorithm \citep{rousseeuw1999fast}, the k-Nearest Neighbors (kNN) approach \citep{ramaswamy2000efficient}, and the contrastive learning–based method proposed by \citet{shenkar2022anomaly}.
Outlier detection constitutes an essential component across scientific research and industrial applications. \\
The presence of anomalous data points can compromise statistical inference. 
Specifically, outliers possess the capacity to inflate estimates of variability (e.g. variance and standard deviation), consequently widening confidence intervals. 
This effect increases the probability of committing a Type II error (failure to reject a false null hypothesis). 
Conversely, in certain domains, the outliers themselves represent the primary interest. 
Prominent examples include the application in financial fraud detection \citep{ramaswamy2000efficient} and in medical diagnostics for identifying pathological conditions, such as cervical cancer \citep{ijaz2020data}. \\
The primary objective of this investigation is to systematically evaluate the relative efficacy of the three selected algorithms in identifying outliers within a designated dataset.
Following the formal definition and characterization of outliers, the study will detail the underlying mechanisms of the three analyzed methods. 
This paper will begin with the robust statistical method of FAST-MCD. 
Subsequently, the analysis will proceed to discuss the distance-based method of kNN. 
Finally, the Intercont approach, a novel machine learning method will be presented.
Different performance metrics will be discussed to supply a optimal view of the detection performance. 
The empirical investigation will utilize financial datasets, drawing among other things primary research literature.
The study is guided by the following formal hypotheses regarding detection efficacy:
The kNN algorithm is hypothesized to yield the highest anomaly detection performance among the three candidates.
The contrastive learning–based Intercont algorithm is anticipated to outperform the FAST-MCD method but is expected to demonstrate lower efficacy than the kNN algorithm.

\newpage
\section {Einleitung}
\section {Definition Outlier Analysis}

What is the Problem?

What are Outliers?
(S20 S.2)	Outliers are sometimes useless and sometimes do they provide usefull information.
			Outliers deviate extremely from well defined norms.
			In some cases we would rather remove them as they mislead our analysis, while in other cases they could 
			they could be very useful and keeping them would be the best solution.
			E.g. Measurment or recoding errors. exceptional but true values, mis-reporting, sampling error
(H80 S.1)	"An observation wich deviates so much from other observations as to arouse that it was generated by a different mechanism"
(C09 S.1)	Outliers and Anomalies are often been used interchangeably

What is Noise?
(S20 S.2)	Noisy data is data wich we have no benefit. Data that carries only meaningless information
			E.g. Incorrect data type (string type entered for a numeric attrtiute), Erroneaous data values
			Missing values (unrecorded data for an attribute)
(C09 S.3)	Noise removal is driven by the need to remove the unwanted objects before any data analysis is performed

What is noise accommodation?


Are there different outlier types?
(C09 S.7)	Point Anomalities: Simplest type of anomaly. 
			If an indivisual data instance can be considered anomalous with respect to the rest of data, 
			then the instance is termed a point anomaly.


(C09 S.7-8)	Contextual Anomalies also conditional (S07 S?): If a data instance is anomalous in a specific context, but not otherwise.
			Each data instance is defined using two sets of attributes: 
			contextual attributes: are used to determine the context (neighborhood) fo that instance.
			E.g. in times series data, time is a contextual attribute that determines the position of an instance
			on the entire sequence
			Behavioral attributes: define the noncontextual characteristics of an instance. 
			In spartial data sets of rainfall of the entire world, the amoutn of rainfall at any location is behavioral attributes
			A Data instance might be a contextual anomaly in a given context, but an identical data instance could be considered normal in a different instance.
			E.g. A similar example can be found in the credit card fraud detection domain. A con-
			textual attribute in the credit card domain can be the time of purchase. Suppose an
			individual usually has a weekly shopping bill of 100 except during the Christmas
			week, when it reaches 1000. A new purchase of 1000 in a week in July will be con-
			sidered a contextual anomaly, since it does not conform to the normal behavior of the
			individual in the context of time even though the same amount spent during Christmas
			week will be considered normal


(C09 S.8-9)	Collective Anomalities: The individual data instance in a collective anomyl may not be anomaliess by themselves, but their 
			occurance together as a collection is anomalous. Is a collection of related data instances is anomalous wih 
			respect to the entire data set. It can only occur in data sets in wich data instances are related.

			A point anomaly or a collective anomaly can also be a contextual anomaly if
			analyzed with respect to a context. Thus a point anomaly detection problem or collec-
			tive anomaly detection problem can be transformed to a contextual anomaly detection
			problem by incorporating the context information.


What is novelty detection?
(C09 S.3)	wich aims at detectiting previously unobserved patterns in the data, for example, a new topic of discussion in a news group
			The difference between novel patterns and anomalies is that the novel patterns are typically incorporated 
			into the normal model after being detected. 
			Solutions for these related problems are often used for anomaly detection and novelty detection

What is outlier explanation?
(P22 s.978)	In fields where ouliers do have meaningfull information, it is necessary to generate an importance level 
			of outliers bescause users have make decisions based of prioritization
(P22 S. 983f)e.g. Fraud detection: Credit Card fraud detection deals with millions of transactions. 
			To reduce false-positive detection ad the costs that come with it, humans analysts are monitoring 
			all suspicious transactions.
			If those suspicious transactions would be prioritized, it could lead to a more efficient result. 
How to deal with outliers
\todo{Test Test Test Test Test}
\clearpage
\bibliographystyle{autorjahrdidiDE}  % Dies ist der verwendete Stil f�r Zitation und Literaturverzeichnis																	 % Die Datei autorjahrdidiDE.bst muss in das aktuelle Arbeitsverzeichnis kopiert werden
\bibliography{Literatur}	           % Literaturverzeichnis liegt in der Datei Literatur
\end{document}											 % Ende des Dokuments
