\documentclass[a4paper, 	% Seitenformat
		12pt, 								% Schriftgr��e
		bibliography=totoc, 	% Literaturverzeichnis in das Inhaltsverzeichnis
		index=totoc, 	% Index in das Inhaltsverzeichnis
		abstract=true, 	% mit Abstrakt
		headsepline, 	% Trennlinie f�r die Kopfzeile
		%footnosepline,	% Trennlinie f�r die Fusszeile
		]{scrartcl}

\usepackage[T1]{fontenc}
\usepackage[utf8]{inputenc}
\usepackage{lmodern}
\usepackage[ngerman]{babel}
%\usepackage{ulem}
% Bilder
%\usepackage{pst-all}		% Zeichnungen in Latex (kein pdflatex)
\usepackage{subfig}	  	% Unterteilungen von Abbildungen
\usepackage{float}		  % Gleitobjekte
\usepackage{graphicx}		% Einbinden von Graphiken aus Dateien
\usepackage{rotating}   % Die folgenden Packages sind meist hilfreich f�r das Erstellen von Arbeiten
\usepackage{mathrsfs}
\usepackage{eurosym,bm,amsmath,amssymb,verbatim}
\usepackage{listings}
\usepackage[T1]{fontenc}
\usepackage{setspace}
\usepackage [round,authoryear] {natbib} % Notwendiges Package f�r den verwendeten Literaturstil
%\usepackage{natbib}





%------------------------------------------------------------------------

\parindent0em 			% kein Einzug nach einer Leerzeile
\topmargin-20.4mm   % F�r die R�nder
\textheight245mm
\textwidth157mm \oddsidemargin11mm \evensidemargin11mm
\hyphenpenalty=100
\exhyphenpenalty=100

%%%%%%%%%%%%%%%%%%%%%%%%%%%%%%%%%%%%%%%%%%%%%%%%%%%%%%%%%%%%%%%%%%%%%%%
%%%%%%%%%%%%%%%%%%%%%%%%%%%%%%%%%%%%%%%%%%%%%%%%%%%%%%%%%%%%%%%%%%%%%%%
%%%%%%%%%%%%%%%%%%%%%%%%%%%%%%%%%%%%%%%%%%%%%%%%%%%%%%%%%%%%%%%%%%%%%%%

\begin{document}					% Start des Dokuments
\begin{titlepage}

\hspace{12cm}\scalebox{0.3}{\includegraphics{Unilogo.jpg}}

\vspace*{1cm}
\large\textbf{Disposition: }\\[0.5cm]
\large\textbf{From Statistics to Machine Learning: \\ A Comparative Study on Outlier Detection \\ in multivariate Distributions}\\[0.5cm]

Studiengang: Betriebswirtschaftslehre\\

Lehrstuhl für Statistik\\[1.5cm]
%
%
\noindent\begin{tabular}[h]{@{}ll}
%\begin{tabular}{l l}
Eingereicht bei: &Prof. Dr. Yarema Okhrin \\
\\
Betreuer: &Herr Philipp Haid 			 \\
\\
Vorgelegt von:   &Thore Johannsen         \\
\\
Adresse: &Salomon-Idler-Straße 4         \\[0.2cm]
                 &Augsburg                      \\
\\
Matrikel-Nr.:    &2231165                 \\
\\
Email:           &thore.johannsen@outlook.com     \\
\\
\end{tabular}

\vfill
Augsburg, im September 2010

\end{titlepage}


\newpage
\pagenumbering{Roman}


%\tableofcontents		    % Inhaltsverzeichnis
%\newpage
%\listoffigures			    % Abbildungsverzeichnis
%\listoftables				  % Tabellenverzeichnis

\clearpage
\pagenumbering{arabic}
\onehalfspacing 			     % F�r richtigen Zeilenabstand

%%%%%%%%%%%%%%%%%%%%%%%%%%%%%%%%%%%%%%%%%%%%%%%%%%%%%%%%%%%%%%%%%%%%%%%
%%%%%%%%%%%%%%%%%%%%%%%%%%%%%%%%%%%%%%%%%%%%%%%%%%%%%%%%%%%%%%%%%%%%%%%
%%%%%%%%%%%%%%%%%%%%%%%%%%%%%%%%%%%%%%%%%%%%%%%%%%%%%%%%%%%%%%%%%%%%%%%

%\input{Kapitel/kapitel1}  % Zur besseren Strukturierung empfiehlt es sich das Dokument aufzuteilen.
%\newpage
%\input{Kapitel/kapitel2}
%\newpage
%\input{Kapitel/kapitel3}
%\newpage
%\input{Kapitel/kapitel4}
%\newpage
%\input{Kapitel/kapitel5}
%\vspace{5em}

%%%%%%%%%%%%%%%%%%%%%%%%%%%%%%%%%%%%%%%%%%%%%%%%%%%%%%%%%%%%%%%%%%%%%%%
%%%%%%%%%%%%%%%%%%%%%%%%%%%%%%%%%%%%%%%%%%%%%%%%%%%%%%%%%%%%%%%%%%%%%%%
%%%%%%%%%%%%%%%%%%%%%%%%%%%%%%%%%%%%%%%%%%%%%%%%%%%%%%%%%%%%%%%%%%%%%%%

This thesis undertakes a comparative performance analysis of three outlier detection methods: the FAST-MCD algorithm \citep{rousseeuw1999fast}, the k-Nearest Neighbors (kNN) approach \citep{ramaswamy2000efficient}, and the contrastive learning–based method proposed by \citet{shenkar2022anomaly}.
Outlier detection constitutes an essential component across scientific research and industrial applications. \\
The presence of anomalous data points can compromise statistical inference. 
Specifically, outliers possess the capacity to inflate estimates of variability (e.g. variance and standard deviation), consequently widening confidence intervals. 
This effect increases the probability of committing a Type II error (failure to reject a false null hypothesis). 
Conversely, in certain domains, the outliers themselves represent the primary interest. 
Prominent examples include the application in financial fraud detection \citep{ramaswamy2000efficient} and in medical diagnostics for identifying pathological conditions, such as cervical cancer \citep{ijaz2020data}. \\
The primary objective of this investigation is to systematically evaluate the relative efficacy of the three selected algorithms in identifying outliers within a designated dataset.
Following the formal definition and characterization of outliers, the study will detail the underlying mechanisms of the three analyzed methods. 
This paper will begin with the robust statistical method of FAST-MCD. 
Subsequently, the analysis will proceed to discuss the distance-based method of kNN. 
Finally, the Intercont approach, a novel machine learning method will be presented.
Different performance metrics will be discussed to supply a optimal view of the detection performance. 
The empirical investigation will utilize financial datasets, drawing among other things primary research literature.
The study is guided by the following formal hypotheses regarding detection efficacy:
The kNN algorithm is hypothesized to yield the highest anomaly detection performance among the three candidates.
The contrastive learning–based Intercont algorithm is anticipated to outperform the FAST-MCD method but is expected to demonstrate lower efficacy than the kNN algorithm.


\clearpage

\bibliographystyle{autorjahrdidiDE}  % Dies ist der verwendete Stil f�r Zitation und Literaturverzeichnis
																		 % Die Datei autorjahrdidiDE.bst muss in das aktuelle Arbeitsverzeichnis kopiert werden
\bibliography{Literatur}	           % Literaturverzeichnis liegt in der Datei Literatur

\end{document}											 % Ende des Dokuments
